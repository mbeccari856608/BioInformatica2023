\documentclass{article}
\usepackage{tikz}
\usetikzlibrary{graphs,graphdrawing}
\usepackage{caption}
\usepackage[smartEllipses]{markdown}
\usepackage{hyperref}
\hypersetup{
	colorlinks=true,
	linkcolor=blue,
	filecolor=magenta,      
	urlcolor=blue,
}
\usepackage{multirow}
\usepackage{multicol}
\usepackage[framemethod=tikz]{mdframed}
\usepackage{lipsum}
\usepackage{tkz-graph}
\usegdlibrary{trees}
\usepackage{amsmath, amssymb}
\usepackage{pst-node}
\usepackage{blkarray}
\usepackage{amsmath, amssymb}
\usepackage{pst-node}

\usepackage[margin=2cm]{geometry}

\usepackage[utf8]{inputenc}
\begin{document}
	{Michele Beccari 856608 - BioInformatica -  Progetto spaced seeds - Giugno 2023} 
	
	\section{Progetto Spaced Seeds}

Lo scopo del progetto è individuare uno SNP o un singolo errore su una read $r$ rispetto al reference $R$.

\begin{itemize}
	\item \textbf{INPUT}: due sequenze $r$ e $R$
	\item \textbf{OUTPUT}: la posizione (se esiste) in $r$ di uno SNP oppure di un errore, ovvero la posizione in cui $r$ differisce per una sottostringa $r'$ da $R$ da cui $r$ può essere derivata. 
\end{itemize}

\begin{mdframed}[hidealllines=true,backgroundcolor=blue!20]
Esempio: dato $R$ =AAAAAGGGG e $r$=AACGG, chiaramente $r$ differisce in posizione 3 da $r'$ = AAGGG a causa dello SNP C che sostituisce G.
\end{mdframed} 

\subsection{Metodo proposto}
Una possibile soluzione è utilizzare gli \textbf{spaced seeds}

\begin{mdframed}[hidealllines=true,backgroundcolor=blue!20]
Uno spaced seed consiste in un k-mer dove alcune posizioni sono indicate come * che sta per “do not care”.
\end{mdframed} 

Ad esempio il k-mer AA*GG indica che al posti di * possiamo mettere qualunque simbolo.

Uno spaced seed può matchare diversi k-meri di R. \\
Ad esempio AA*GG matcha AAAGG oppure AAGGG.

\begin{mdframed}[hidealllines=true,backgroundcolor=blue!20]
Definiamo un k-mer 1-approssimato se solo una posizione del k-mer è *.
\end{mdframed}

Abbiamo diversi k-mer 1-approsimati per le varie posizioni da 1 a k.


\subsection{Implementazione}

Utilizzando  \href{https://github.com/bcgsc/ntHash}{ntHash2} è possibile indicizzare la stringa R con k-mer 1-approssimati, distinguendo l'hashing per * in posizione $1,2,3...k$


Ipotizzando che per ogni posizione $i$ di $r$ ci sia uno SNP o un errore si può utilizzare il risultato precedente per capire se saltando una determinata posiziione $i$ di $r$ si ottiene un match con i k-mers memorizzati per R.


\end{document}



