\documentclass{article}
\usepackage{tikz}
\usetikzlibrary{graphs,graphdrawing}
\usepackage{caption}
\usepackage[smartEllipses]{markdown}
\usepackage{hyperref}
\hypersetup{
	colorlinks=true,
	linkcolor=blue,
	filecolor=magenta,      
	urlcolor=blue,
}
\usepackage{multirow}
\usepackage{multicol}
\usepackage[framemethod=tikz]{mdframed}
\usepackage{lipsum}
\usepackage{tkz-graph}
\usegdlibrary{trees}
\usepackage{amsmath, amssymb}
\usepackage{pst-node}
\usepackage{blkarray}
\usepackage{amsmath, amssymb}
\usepackage{pst-node}
\usepackage{listings}
\usepackage{xcolor}
\lstset{language=C++,
	basicstyle=\ttfamily,
	keywordstyle=\color{blue}\ttfamily,
	stringstyle=\color{red}\ttfamily,
	commentstyle=\color{green}\ttfamily,
	morecomment=[l][\color{magenta}]{\#}
}

\usepackage[margin=2cm]{geometry}

\usepackage[utf8]{inputenc}
\begin{document}
	{Michele Beccari 856608 - BioInformatica -  Progetto spaced seeds - Giugno 2023} 
	
	\section{Progetto Spaced Seeds}

Lo scopo del progetto è individuare uno SNP o un singolo errore su una read $r$ rispetto al reference $R$.

\begin{itemize}
	\item \textbf{INPUT}: due sequenze $r$ e $R$.
	$r$ differisce di uno SNP o errore da una sottostringa di R, cioe $\exists \; r'$ sottostringa di $r$ tale che $hamming(r, r')$ = 1
	\item \textbf{OUTPUT}: la posizione (se esiste) in $r$ di uno SNP oppure di un errore, ovvero la posizione in cui $r$ differisce per una sottostringa $r'$ da $R$ da cui $r$ può essere derivata. 
\end{itemize}

\begin{mdframed}[hidealllines=true,backgroundcolor=blue!20]
Esempio: dato $R$ =AAAAATCGG e $r$=ATAGG, chiaramente $r$ differisce in posizione 2 da $r'$ = ATCGG a causa dello SNP C che sostituisce A.
\end{mdframed} 

\subsection{Metodo proposto}
Una possibile soluzione è utilizzare gli \textbf{spaced seeds}

\begin{mdframed}[hidealllines=true,backgroundcolor=blue!20]
Uno spaced seed consiste in un k-mer dove alcune posizioni sono indicate come * che sta per “do not care”.
\end{mdframed} 

Ad esempio il k-mer AA*GG indica che al posti di * possiamo mettere qualunque simbolo.

Uno spaced seed può matchare diversi k-meri di R. \\
Ad esempio AA*GG matcha AAAGG oppure AAGGG.

\begin{mdframed}[hidealllines=true,backgroundcolor=blue!20]
Definiamo un k-mer 1-approssimato se solo una posizione del k-mer è *.
\end{mdframed}

Abbiamo diversi k-mer 1-approsimati per le varie posizioni da 1 a k.

Ipotizzando che per ogni posizione $i$ di $r$ ci sia uno SNP si può capire se saltando una determinata posizione $i$ di $r$ si ottiene un match con i k-mers memorizzati per R.

\subsection{Implementazione}

Utilizzando  \href{https://github.com/bcgsc/ntHash}{ntHash2} è possibile indicizzare la stringa R con k-mer 1-approssimati, distinguendo l'hashing per * in posizione $1,2,3...k$


L'idea è quella di:

\begin{enumerate} 
 \item	Ottenere tutti k-mer 1-approssimati della stringa $r$.
 \item  Per ogni k-mer 1-approssimato nella stringa $R$ eseguire un confronto con i k-mer 1-approssimati trovati al punto 1.
 \item Quando eventualmente si trova una match, verificare il carattere corrispondente alla posizione dell' "*" nella stringa $r$ per trovare lo SNP.
\end{enumerate}

Per generare gli hash dei k-mer 1-approssimati la libreria offre degli oggetti appositi:

\begin{lstlisting}
	   // Oggetto che genera un hash
	   // per i k-meri di lunghezza 5 della  stringa "ATAGG", utilizzando
	   // i seed forniti.
	   // Gli hash sono degli interi senza segno a 64 bit
	   nthash::SeedNtHash("ATAGG", seeds, 1, 5)
\end{lstlisting}

Per generare gli hash con un "*" in una determinata posizione è necessario fornire all'oggetto della libreria uno o più \textit{seed},  ovvero una stringa composta da 0 e 1 dove gli 0 rappresentano gli asterischi.

Potenzialmente è possibile generare anche più di un hash per ogni sequenza per gestire eventuali conflitti.




\end{document}



